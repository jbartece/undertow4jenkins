%=========================================================================
%Jakub Bartecek (xbarte09)
%Date: 2013 - 2014
%Encoding: UTF-8
%set syntax=tex




\chapter{Úvod}
    Tento semestrální projekt se zabývá vylepšením systému Jenkins CI, který je v praxi velmi využíván pro potřeby průběžného testování softwaru
    a jeho kontinuální integraci. Vylepšení se týká především webového serveru a \emph{servlet} kontejneru, který je v Jenkins integrován. 
    V současném stavu tyto funkce vykonává kombinace nástrojů Winstone a Jetty. 
    
    Server Winstone je již neudržovaný a zastaralý nástroj a z tohoto důvodu
    byl z velké části nahrazen serverem Jetty, který potřebnou funkcionalitu poskytuje. Server Jetty je poměrně komplexní projekt a nabízí
    mnoho funkcionality, ale na druhou stranu jeho rozsah nedovoluje poskytovat maximální rychlost.
    
    V současné době vznikl nový webový server Undertow, který si klade za cíl být co nejjednodušší a nejrychlejší a mohl by být přínosný
    a vhodný pro systém Jenkins. Jelikož tento server je nový a je sponzorován firmou Red Hat, tak lze předpokládat, že jeho vývoj
    bude nadále pokračovat a nebude zastarávat.

    Cílem této práce je nahradit server Jetty a případně i serveu Winstone pomocí serveru Undertow 
    a integrovat jej se systémem Jenkins CI. Při integraci je kladen důraz na snahu
    zachovat zpětnou kompatibilitu se starým řešením. 

    V rámci tohoto semestrálního projektu jsou nejprve rozebrány potřebné informace týkající se jednotlivých nástrojů a plánované integrace.
    Následně je detailně analyzována architektura systému Jenkins a způsob jeho integrace se servery Winstone a Jetty. V následující části
    je diskutována varianta nahrazení pouze serveru Jetty a varianta nahrazení serveru Jetty i Winstone pomocí serveru Undertow. 
    Z provedené analýzy je zvolena jedna varianta, která je vybrána pro následnou integraci.

    Samotná implementace integrace není předmětem tohoto semestrálního projektu a bude provedena až v navazujícím diplomovém projektu. 
    V rámci dimplomového projektu bude také provedeno testování výkonu modifikovaného systému Jenkins CI a celkové shodnocení navržených
    a provedených změn.



\chapter{Teoretická část}
    \section {Jenkins}
        Obecné informace o Jenkins a asi i nějaké detaily - k čemu slouží, jeho přínos, apod.
        kniha o Jenkins \cite{JenkinsBook}.
        
        Typy spouštění a překlady, spouštění testů

        Co je servlet kontejner?

    \section{Maven}
        Základní obecné informace, něco o pomkách, způsobech překladů

    \section{Winstone a Jetty}
        Aspoň základní informace, aby následující kapitola byla lehčeji pochopitelná?

    \section{Undertow}
        Obecné informace, k čemu je dobrý

        Jak se používá? Jak moc do detailů?

        


\chapter{Analýza problému}
    
    \section{Aktuální stav arhitektury servlet kontejneru v Jenkins}
        Popis aktuálního stavu architektury Jenkins z pohledu servlet kontejneru, napojení Winstone a Jetty v Jenkins
        extras-executeblas-war, vkládání winstonu jako .jar

    \section{Zjištěné problémy}
        Rozdílná verze Javy - Undertow v. 7, Jenkins v. 6
        
        
    \section{Varianta ponechání Winstonu}

    \section{Varianta nahrazení Jetty i Winstonu}

    \section{Zvolená varianta}
        Způsob vyhodnocení správnosti výsledného řešení pomocí testů jenkins


\chapter{Otázky}
    Zveřejnit odkazy na repozitáře v práci?

\chapter{Závěr}


%=========================================================================



